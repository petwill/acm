\documentclass[10pt,twocolumn,oneside]{article}
\setlength{\columnsep}{10pt}                                                                    %兩欄模式的間距
\setlength{\columnseprule}{0pt}                                                                %兩欄模式間格線粗細



\usepackage{fontspec}                               %設定字體
\usepackage{color}
\usepackage[x11names]{xcolor}

\usepackage{xeCJK}                              %xeCJK
\setCJKmainfont{AR PL UMing CN}
\usepackage{listings} 
\usepackage{fancyhdr} 

\usepackage{amsmath, courier, listings, fancyhdr, graphicx}
\topmargin=0pt
\headsep=5pt
\textheight=780pt
\footskip=0pt
\voffset=-40pt
\textwidth=545pt
\marginparsep=0pt
\marginparwidth=0pt
\marginparpush=0pt
\oddsidemargin=0pt
\evensidemargin=0pt
\hoffset=-42pt

\lstset {                                            % Code顯示
language=C++,                                       % the language of the code
basicstyle=\footnotesize\ttfamily,                      % the size of the fonts that are used for the code
%numbers=left,                                      % where to put the line-numbers
numberstyle=\footnotesize,                      % the size of the fonts that are used for the line-numbers
stepnumber=1,                                       % the step between two line-numbers. If it's 1, each line  will be numbered
numbersep=5pt,                                      % how far the line-numbers are from the code
backgroundcolor=\color{white},                  % choose the background color. You must add \usepackage{color}
showspaces=false,                                   % show spaces adding particular underscores
showstringspaces=false,                         % underline spaces within strings
showtabs=false,                                 % show tabs within strings adding particular underscores
frame=false,                                            % adds a frame around the code
tabsize=2,                                          % sets default tabsize to 2 spaces
captionpos=b,                                       % sets the caption-position to bottom
breaklines=true,                                    % sets automatic line breaking
breakatwhitespace=false,                            % sets if automatic breaks should only happen at whitespace
escapeinside={\%*}{*)},                         % if you want to add a comment within your code
morekeywords={*},                                   % if you want to add more keywords to the set
keywordstyle=\bfseries\color{Blue1},
commentstyle=\itshape\color{Red4},
stringstyle=\itshape\color{Green4},
}

\begin{document}
\pagestyle{fancy}
\fancyfoot{}
\fancyfoot[R]{\includegraphics[width=20pt]{ironwood.jpg}}
\fancyhead[L]{National Taiwan University}
\fancyhead[R]{\thepage}
\renewcommand{\headrulewidth}{0.4pt}
\renewcommand{\contentsname}{Contents} 


%\scriptsize
\tableofcontents
%%%%%%%%%%%%%%%%%%%%%%%%%%%%%

\newpage

\section{Basic}
\subsection{.vimrc}
\lstinputlisting{_vimrc}
\subsection{Increase Stack Size}
\lstinputlisting{IncStack.cpp}
\subsection{digitDP}
\lstinputlisting{DP/digitDP.cpp}
\subsection{DP(convex hull optimization)}
\lstinputlisting{DP/convexHullDP.cpp}
\subsection{simulated annealing}
\lstinputlisting{Esc/simAnneal.cpp}


\section{Graph}
\subsection{HLD}
\lstinputlisting{graph/HLD/template.cpp}
\subsection{Hungarian}
\lstinputlisting{graph/bipartite_matching/hungarian.cpp}
\subsection{KM}
\lstinputlisting{graph/bipartite_matching/km.cpp}
\subsection{Bi-vertex-connected Subgraph}
\lstinputlisting{graph/tarjan/割點and點雙連通分量/template.cpp}
\subsection{Bi-edge-connected Subgraph}
\lstinputlisting{graph/tarjan/橋and邊雙連通分量/template.cpp}
\subsection{SCC}
\lstinputlisting{graph/tarjan/SCC/SCC(tarjan).cpp}
\subsection{Steiner Tree( PECaveros )}
\lstinputlisting{graph/steinerTree.cpp}
\subsection{Edmond's Matching Algorithm}
\lstinputlisting{graph/blossom2.cpp}
\subsection{Tree Decomposition}
\lstinputlisting{graph/treeDecomposition.cpp}
\subsection{Tree Longest Path}
\lstinputlisting{graph/treeLongestPath.cpp}

\section{Flow}
\subsection{Dinic Maxflow}
\lstinputlisting{flow/dinic.cpp}

\section{Data Structure}
\subsection{Disjoint Set}
\lstinputlisting{dataStructure/djs.cpp}
\subsection{Djs + Seg}
\lstinputlisting{dataStructure/djs+seg.cpp}
\subsection{Sparse Table}
\lstinputlisting{dataStructure/sparseTable.cpp}
\subsection{Link Cut Tree}
\lstinputlisting{dataStructure/linkCutTree.cpp}
\subsection{Treap}
\lstinputlisting{dataStructure/treap/template.cpp}

\section{Math}
\subsection{Prime Table}
\lstinputlisting{math/primeTable.cpp}
\subsection{Miller Rabin Prime Test}
\lstinputlisting{math/millerRabin.cpp}
\subsection{Extended Euclidean Algorithm}
\lstinputlisting{math/extgcd.cpp}
\subsection{Gauss Elimination}
\lstinputlisting{math/gaussElimination.cpp}
\subsection{FFT}
\lstinputlisting{math/fft.cpp}
\lstinputlisting{math/fftMod.cpp}
\subsection{NNT}
\lstinputlisting{math/ntt.cpp}
\lstinputlisting{math/nttCrt.cpp}
\subsection{Big Number}
\lstinputlisting{math/bigNum.cpp}

\section{string}
\subsection{Palindromic Tree}
\lstinputlisting{string/palindromicTree.cpp}
\subsection{Suffix Array}
\lstinputlisting{string/suffix_array.cpp}
\subsection{Longest Palindromic Substring}
\lstinputlisting{string/longest_palindromic_substring.cpp}

\section{geometry}
\subsection{Point Class}
\lstinputlisting{geometry/PointClass.cpp}
\subsection{Intersection of Circles/Lines/Segments}
//PECaveros
\lstinputlisting{geometry/interCircle.cpp}
\lstinputlisting{geometry/interLine.cpp}
\lstinputlisting{geometry/interSeg.cpp}

\subsection{Convex Hull}
\lstinputlisting{geometry/findCenter+convexHull.cpp}
\subsection{Half Plane Intersection}
\lstinputlisting{geometry/half_plane_intersection.cpp}

\end{document}
